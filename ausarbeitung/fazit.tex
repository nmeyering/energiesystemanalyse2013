Zusammenfassend kann gesagt werden, dass im Bereich der Technologie ein großes Innovationspotential vorliegt. In den Bereichen Kollektoren, Speichertechnologien und Transferflüssigkeiten ist durch neue Technologien eine Effizienzsteigerung zu erwarten. Außerdem sind in Zukunft durch derzeit noch in der Entwicklung befindliche Konzepte wie Fresnel-Technologie und Solarturmkraftwerke~\cite{irena2012} ggf. noch größere Einsparungen möglich.

Damit die CSP-Technologie auf dem Markt wettbewerbsfähig werden kann, ist es erforderlich, dass Kosten deutlich sinken. Die dafür notwendigen Rahmenbedingungen beinhalten verstärkte Anreize aus der Politik und Investition in Forschung und Entwicklung.

In der Anschaffung entstehen die meisten Kosten für CSP-Kraftwerke (etwa 80\,\%). Wartung und Betrieb sind im Vergleich sehr günstig. Aktuelle Szenarien sehen Kostensenkung von etwa 40\,\% bis 2020 vor. Die Lernraten sind zur Zeit wegen mangelnder installierter Kapazität noch schwer mit Sicherheit zu benennen, Schätzungen liegen aber bei etwa 10\,\%.

Eine Akzeptanzfrage für Deutschland stellt sich eher nicht, da es hierzulande keine Standorte geplant oder sinnvoll sind, sondern ausschließlich im "`sun belt"'.

Die Ökobilanz für CSP-Technologie ist im Hinblick auf THG-Emissionen vergleichbar mit anderen Erneuerbaren Energien mit um die 30\,g CO\textsubscript{2}-Äquivalente/kWh\textsubscript{el}. Im Gegensatz zu einigen Photovoltaik-Technologien werden keine seltenen Rohstoffe benötigt. Die Ökologische Verträglichkeit ist nach aktuellem Erkenntnisstand insgesamt sehr gut.

Die Wirtschaftliche Bedeutung in Deutschland ist einerseits über den Technologieexport interessant - es ist bereits viel technisches Know-How vorhanden.
Andererseits stellt Strom aus CSP-Erzeugung über Importe einen möglichen zukünftigen Anteil am deutschen erneuerbaren Energiemix dar.

