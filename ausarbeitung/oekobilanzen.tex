Die ökologischen Auswirkungen, die durch den Einsatz von Solarthermischen Kraftwerken entstehen, können durch eine Ökobilanz oder \emph{Life Cycle Assessment} (LCA) abgeschätzt werden.
In der Literatur finden sich dazu bisher relativ wenige Analysen, was vermutlich der bisher eher geringen wirtschaftlichen Bedeutung dieser Technik zuzuschreiben ist. Die vorhandenen Quellen, die sich mit LCA beschäftigen, untersuchen insbesondere die Größen Produktion von Treibhausgasen (THG), Wasser- und den Energieverbrauch.
Dabei werden üblicherweise die folgenden drei Kategorien der Lebensdauer einer Anlage in Betracht gezogen:
\begin{enumerate}
\item Herstellung. Dies beinhaltet die Gewinnung von Rohmaterialien und Zwischenschritte über sämtliche Zulieferketten
\item Betrieb. Hierin sind u.a. Instandhaltung und ggf. Treibstoffverbrauch sowie Stromverbrauch der Anlage enthalten
\item Demontage und Entsorgung, also Kosten durch Abbau und Abtransport von Komponenten, Recycling und Lagerungskosten
\end{enumerate}

Eine Studie\cite{viebahn2011} von Viebahn et al.\ aus dem Jahre 2011 untersucht u.a.\ die Entwicklung der möglichen THG-Emissionen durch den Ausbau von Solar CSP-Technik in den drei verschiedenen IEA-Szenarien\cite{iea2010} bis 2050.
Für zwei Beispieltechnologien aus Spanien bzw. Algerien belaufen sich die aktuellen (Stand 2010) Emissionen im Mittel auf etwas über 30\,g CO\textsubscript{2}-Äquivalente/kWh\textsubscript{el} und es wird eine Reduktion auf knapp unter 20\,g CO\textsubscript{2}-Äquivalente/kWh\textsubscript{el} für das Szenario 2050 errechnet.
Diese Emissionswerte stammen zu etwa 90\,\% aus der Herstellung.

Burkhardt et al.\ (2011)\cite{burkhardt2011} kommen bei der LCA einer konkreten Anlage in Kalifornien zu ähnlichen Werten für die THG-Belastung von insgesamt 26 bis 28\,g CO\textsubscript{2}-Äquivalente/kWh\textsubscript{el}.

Damit liegt die Solarthermie bei der Produktion von THG etwa im Bereich anderer Erneuerbarer Energien wie Photovoltaik oder Windkraft.\cite{viebahn2011}

Andere ökologische Risiken wie Eutrophierung von Gewässern, Versäuerung oder Humantoxizität sind schlechter untersucht und belegt. Eine spanische Studie\nocite{lechon2008} (2008) untersucht am Beispiel einer hybriden Solarthermianlage zusätzlich zu Treibhausgasemissionen neun weitere ökologische Faktoren und kommt zu dem Schluss, dass die Auswirkungen im Vergleich mit den bestehenden Techniken zur fossilen Stromerzeugung gering sind und hauptsächlich durch Stromverbrauch der Hybridanlage selbst entstehen, der wiederum aus fossilen Energien erzeugt wird.

Daten zur LCA in Bezug auf Rohstoff-Ressourcenverbrauch, insbesondere im Vergleich mit konkurrierenden Techniken, ließen sich kaum finden. Es lässt sich aber vielleicht spekulieren, dass zumindest der Bedarf an knappen Rohstoffen eher gering einzuschätzen ist.
