Die zuk�nftige Entwicklung von CSP l�sst sich in zwei Phasen kategorisieren. Zuerst muss eine kommerzieller Anreiz und eine Wettbewerbsf�higkeit auf dem Energiemarkt geschaffen werden. Sollte dies geschehen sein, werden in der zweiten Phase Investoren bereit sein in neue Anlagen zu investieren, was zu einer erh�hten Kapazit�t f�hrt und somit eine Senkung der Kosten mit sich tr�gt. Die zweite Phase ist unter normalen Umst�nden eine Folge einer erfolgreichen Etablierung der Technologie aus Phase eins. Somit liegt der Schl�ssel f�r eine positive Entwicklung in Phase eins. Um einen kommerziellen Anreiz und Wettbewerb f�r CSP zu erlangen ist eine aktive F�rderung der Technologien bis zu einem kritischen Punkt n�tig. Um diesen Punkt zu erreichen sind vor allem die Bereiche von Wichtigkeit, die die �konomischen Eigenschaften eines CSP Projektes betrachten.
F�r eine Entwicklung f�r die Zukunft wurden von der NEEDS (New Energy Externalities Developments for Sustainability) drei verschiedene Szenarien unter Ber�cksichtigung von verschiedenen Konditionen aufgestellt. Es wird hierbei zwischen einer �optimistic-realisitc� Variante und zwei extremen Varianten �very optimistic� und �pessimistic� unterschieden. Die Szenarien orientieren sich an dem oben vorgestellten Zwei-Phasen Ansatz.
 
F�r die Szenarien wird eine Liste von Parametern aufgestellt, die f�r jedes Szenario unterschiedliche Gewichte erhalten (abb)
 
Das �very optimistic� Szenario nimmt an, dass in beide Phasen das volle Potential erreicht wird. In der ersten Phase wird also ein fr�her Anstieg von CSP Kapazit�ten erwartet. Hierzu ist ein globale Klimaschutz Abkommen notwendig, welches alle erneuerbaren Energien enth�lt und ein Regelwerk vorsieht.
Zum Vergleich sieht das �optimistic-realistic� Szenario sieht kein unmittelbares Erreichen aller m�glichen Potentiale in Phase eins vor. Es geht davon aus, dass sich diese jedoch im Laufe der Zeit aktivieren. Desweiteren wird nicht davon ausgegangen, dass nukleare und fossile Energien v�llig verdr�ngt werden. Durch Einspeisegesetze, unterst�tzt durch erh�hten Preise f�r nukleare und fossile Energien, wird jedoch Anstieg von CSP prognostiziert.
Der Ansatz f�r das �pessimistic� Szenario sieht eine Verz�gerung der Entwicklung von CSP um weitere Dekaden vor. In Phase eins wird der kritische Punkt nicht erreicht und es wird zu keiner f�rdernden Entwicklungsphase, noch zu einer resultierenden Phase zwei kommen. CSP Anlagen werden nicht v�llig aus den erneuerbaren Energien 