Solarthermie (Concentrated Solar Power (CSP)) ist eine vielversprechende erneuerbare Energiequelle f�r die Zukunft. Das Verfahren zur Energiegewinnung nutzt die Sonnenstrahlung zur Gewinnung von W�rme, um diese dann durch Gas oder Dampfmaschinen in elektrische Energie umzuwandeln. In den letzten Jahren hat die Technik zunehmend an Aufmerksamkeit gewonnen, was Gro�projekten wie Desertec und Euromed best�tigen. Zum jetzigen Zeitpunkt existieren unterschiedliche Technologien zur Nutzung der Sonnenstrahlung. F�r die Zukunft stellen sich also die Fragen, f�r welche Regionen sich eine Technologie auf lange Sicht am besten eignet, welche Kapazit�ten installiert werden m�ssen und wie viel Energie �berhaupt erzeugt werden kann. Desweitern m�ssen ebenfalls Techniken zur Speicherung der Energie bei geringer Nachfrage entwickelt und getestet werden. All diese Punkte stehen nat�rlich in direktem Zusammenhang mit �konomischen und �kologischen Kosten, die es zu analysieren gilt. Die derzeitigen Prognosen zur Marktentwicklung von sehen bis 2015 einen Anteil von etwa 12 GW durch Sorarthermie gewonnen Strom vor.