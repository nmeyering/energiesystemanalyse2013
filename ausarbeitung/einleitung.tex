Solarthermie (Concentrated Solar Power (CSP)) ist eine vielversprechende erneuerbare Energiequelle für die Zukunft. Das Verfahren zur Energiegewinnung nutzt die Sonnenstrahlung zur Gewinnung von Wärme, um diese dann durch Gas oder Dampfmaschinen in elektrische Energie umzuwandeln. In den letzten Jahren hat die Technik zunehmend an Aufmerksamkeit gewonnen, was Großprojekten wie Desertec und Euromed bestätigen. Zum jetzigen Zeitpunkt existieren unterschiedliche Technologien zur Nutzung der Sonnenstrahlung. Für die Zukunft stellen sich also die Fragen, für welche Regionen sich eine Technologie auf lange Sicht am besten eignet, welche Kapazitäten installiert werden müssen und wie viel Energie überhaupt erzeugt werden kann. Desweitern müssen ebenfalls Techniken zur Speicherung der Energie bei geringer Nachfrage entwickelt und getestet werden. All diese Punkte stehen natürlich in direktem Zusammenhang mit ökonomischen und ökologischen Kosten, die es zu analysieren gilt. Die derzeitigen Prognosen zur Marktentwicklung von sehen bis 2015 einen Anteil von etwa 12 GW durch Sorarthermie gewonnen Strom vor.
